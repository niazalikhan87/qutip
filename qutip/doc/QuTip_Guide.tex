\documentclass[11pt,letterpaper]{article} 
\usepackage{amsmath}
\usepackage{listings}
\usepackage{hyperref}
\usepackage{multicol}
\usepackage{color}

\begin{document} 

\title{\textbf{\huge{QuTiP: Quantum Toolbox in Python\\ \phantom{a}}}Version 0.1}

\author{\textbf{Paul D. Nation} \& \textbf{Robert J. Johansson}\\
\phantom{a}\\
Digital Materials Laboratory\\
RIKEN Advanced Science Institute\\
Wakoshi, Saitama, 351-0198 Japan\\
\phantom{a}\\
Copyright (c) 2011}
\date{}
\maketitle
\begin{multicols}{2}
\tableofcontents
\end{multicols}

\section{Getting Started}\label{sec:started}

\subsection{Introduction}\label{subsec:intro}
QuTiP is designed to be an open-source software solution to solve for the dynamics of open quantum systems.  QuTiP is written in the Python programming language, allowing for platform-independent operation without the need  to compile any source code.  The numerics underlying QuTiP are based on the standard high-performance numerical and graphical packages available for Python:
\begin{flushleft}
\hspace{1in}\textbf{NumPy:}\href{http://numpy.scipy.org}{http://numpy.scipy.org}\\
\hspace{1in}\textbf{SciPy:}\href{http://scipy.org}{http://scipy.org}\\
\hspace{1in}\textbf{matplotlib:}\href{http://matplotlib.sourceforge.net}{http://matplotlib.sourceforge.net}\\
\hspace{1in}\textbf{PyGTK:}\href{http://www.pygtk.org}{http://www.pygtk.org} (Linux / Windows)\\
\hspace{1in}\textbf{CocoaDialog:}\href{http://cocoadialog.sourceforge.net}{http://cocoadialog.sourceforge.net} (Mac)
\end{flushleft}

The creation of QuTiP was inspired by the highly successful \textcolor{blue}{qotoolbox} \href{http://www.qo.phy.auckland.ac.nz/qotoolbox.html}{http://www.qo.phy.auckland.ac.nz/qotoolbox.html}, but goes beyond by providing for a more flexible work environment, capable of handling general time-dependent problems.  In addition, QuTip enables support for parallel processing found in most modern computers.

\subsection{Installation}\label{subsec:install}
QuTiP uses the standard Python distribution tool \textcolor{blue}{distutil} for installation.  Once the dependences have been satisfied, installing QutiP is as simple as downloading the installer and running from the command-line:
\begin{lstlisting}[basicstyle=\small,keywordstyle=\color{red}\bfseries,numberstyle=\tiny,firstnumber=last] 
  sudo python setup.py install
\end{lstlisting}
which will install QuTiP into the directory specified by your Python distribution.  Directions for installing the dependences can be found via the links in Sec.~\ref{subsec:intro}.  However, in the the sections that follow, we will briefly highlight the quickest installation method for the various platforms.  

\subsubsection{Linux}\label{subsubsec:linux}
The quickest way to get started using Linux is to install the necessary packages via your distributions software management program.  QuTiP requires the \textcolor{blue}{NumPy},  \textcolor{blue}{SciPy}, and  \textcolor{blue}{matplotlib} packages to run.  In addition,  \textcolor{blue}{PyGTK} is an optional (but recommended) install for graphical output.  Finally, it is useful to access QuTip via the command line, which is provided by the  \textcolor{blue}{iPython} \href{http://ipython.scipy.org}{http://ipython.scipy.org} package. 

\subsubsection{Macintosh}\label{subsubsec:mac}
On the Macintosh, it is recommended that you install the necessary programs using the \textcolor{blue}{SciPy SuperPack} \href{http://stronginference.com/scipy-superpack}{http://stronginference.com/scipy-superpack}, or obtain the all-in-one numerical Python distribution provided by \textcolor{blue}{Enthought} \href{http://www.enthought.com}{http://www.enthought.com}, the developers behind the \textcolor{blue}{NumPy} and \textcolor{blue}{SciPy} packages.  This software is freely available for academic institutions and users, but requires a license fee for commercial use.  Using either of these two methods fulfills all the installation requirements for QuTiP.

\subsubsection{Windows}\label{subsubsec:windows}
Windows sucks


\section{Your First Calculation}
\lstdefinelanguage{Custom}{morekeywords={range},sensitive=False}
\lstloadlanguages{Python,Custom}

\lstset{language=Python,alsolanguage=Custom}


\begin{lstlisting}[basicstyle=\small,keywordstyle=\color{red}\bfseries,numbers=left,numberstyle=\tiny,firstnumber=last] 
for i in range(intmax):
  do nothing
print 'done'
\end{lstlisting} 


\end{document}